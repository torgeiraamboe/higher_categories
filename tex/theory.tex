
\subsection{Motivasjon}

Ofte når noe har prefiksen ``kvasi'' betyr det at de nesten er noe vi kjenner til, eller at de i det minste er relatert til de på en fin måte. 
Så med den logikken burde kvasikategorier enten nesten være kategorier, eller hvertfall være relatert til de på en fin måte. 
Foredraget handler jo også om høyere kategorier, så dette burde kanskje komme inn i bildet. 

Det er litt ambiguitivt hva man mener når man bruker ordet ``høyere'' i sammenheng med kategoriteori. 
Ofte så mener man høyere som i ``høyere dimensjon'', men dette skal vi se at er komplisert. 

Uansett hva vi mener så burde vi først se på objektene vi ønsker å generalisere.

\section{Kategorier}

Det finnes flere ulike måter å definere en kategori på, men vi skal se på tre av de mest kjente. 

\begin{definition}
    En kategori $C$ består av
    \begin{itemize}
        \item en samling med objekter $Ob(C)$,
        \item en samling med morfier $C(X,Y)$ for alle objekter $X$ og $Y$ og
        \item for alle tripler av objekter $X, Y, Z$ en binæroperasjon $\circ : C(Y, Z)\times C(X, Y) \longrightarrow C(X, Z)$ kalt komposisjon,
    \end{itemize}
    slik at
    \begin{itemize}
        \item komposisjon er en assosiativ binæroperasjon
        \item for alle objekter $X$ finnes det en morfi $1_X$, kalt identitetsmorfien, slik at $f \circ 1_X = f$ og $1_x\circ g = g$ for alle morfier $f\in C(X, Y)$ og $g\in C(Y,X)$. 
    \end{itemize}
\end{definition}

\begin{definition}
    En (liten) kategori er gitt av et diagram
    \begin{tikzcd}
        S_1 \arrow[r,shift left=.75ex, "s"] \arrow[r,shift right=.5ex, swap, "t"] & S_0
    \end{tikzcd}
    med en morfi $e:S_0 \longrightarrow S_1$ og komposisjon $c: S_1\times S_1 \longrightarrow S_1$ slik at følgende diagram kommuterer
    
    \begin{center}
    \begin{tikzcd}
        C_0 \arrow[r, "e"] \arrow[rd, "1"'] & C_1 \arrow[d, "t"] \\
                                    & C_0               
    \end{tikzcd}
    \begin{tikzcd}
        C_0 \arrow[r, "e"] \arrow[rd, "1"'] & C_1 \arrow[d, "s"] \\
                                    & C_0               
    \end{tikzcd}
    
    \begin{tikzcd}
        C_1 \times C_1 \arrow[r, "c"] \arrow[d, "p_1"'] & C_1 \arrow[d, "s"] \\
        C_1 \arrow[r, "s"']                             & C_0               
    \end{tikzcd}
    \begin{tikzcd}
        C_1 \times C_1 \arrow[r, "c"] \arrow[d, "p_2"'] & C_1 \arrow[d, "t"] \\
        C_1 \arrow[r, "t"']                             & C_0               
    \end{tikzcd}
    
    \begin{tikzcd}
        C_1 \times C_1\times C_1 \arrow[r, "c\times 1"] \arrow[d, "1\times c"'] & C_1\times C_1 \arrow[d, "c"] \\
        C_1\times C_1 \arrow[r, "c"']                                           & C_1                         
    \end{tikzcd}
    
    \begin{tikzcd}
        C_0\times C_1 \arrow[r, "e\times 1"] \arrow[rd, "p_2"'] & C_1\times C_1 \arrow[d, "c"] & C_1\times C_0 \arrow[l, "1\times e"'] \arrow[ld, "p_1"] \\
                                                        & C_1                          &
    \end{tikzcd}
    \end{center}

\end{definition}    

Her er $S_0$ samlingen av objekter mens $S_1$ er samlingen av morfier. 
Denne prosessen danner i sin generelle form det som kalles en intern kategori. 
Man må i det tilfellet bytte ut kartesisk produkt med fiberprodukt. 
Dette kan defineres i mange tilfeller, men i dette tilfellet der den ambiente kategorien er $Set$ produseres små kategorier. 

Den tredje er kanskje litt mer relevant for dette foredraget, men vi skal ikke se nærmere på den enda. 

\begin{definition}
    En kategori er en simplisiell mengde som tilfredsstiller nerveegenskapen. 
\end{definition}


\section{Generalisering}

Så, hvorfor trenger vi generalisering? 
Det første hintet kommer fra vanlig kategoriteori, i at isomorfi av kategorier virker til å være en alt for streng måte å snakke om at to kategorier er like på. 
Man introduserer dermed naturlige transformasjoner og definerer det man kaller en ekvivalens av kategorier. 
Denne definisjonen gir et mye mer nøyansert og ``riktig'' bilde på hva to like kategorier burde være. 
Naturlige transformasjoner er morfier mellom funktorer, så vi innser at vi har ekstra struktur. 
Kategorien av alle (små) kategorier danner altså noe som er mer enn en kategori, ettersom vi har morfier mellom morfiene våre. 
Vi lærer så at funktorer mellom to kategorier og naturlige transformasjoner også danner en kategori, ofte kalt funktorkategorien. 
Altså har vi da en kategori der alle mengdene av morfier også er kategorier. 
Vi har altså ``en ekstra dimensjon''. 
Vi ser dette ved å tegne diagrammer der naturlige transformasjoner tar opp 2-dimensjonal plass, istedenfor bare en linje. 
Vi kaller denne strukturen en streng\footnote{Vi kommer straks tilbake til hvorfor vi bruker ordet streng i tillegg her.} 2-kategori, fordi høyeste dimensjon er 2.

\begin{center}
\begin{tikzcd}[column sep=1.5cm]
A
  \arrow[r, bend left=40, ""{name=v, below,}]  
  \arrow[r, bend right=40,""{name=s, above,}] 
  \arrow[from=v,to=s]
& B
\end{tikzcd}
\end{center}

Dette kan vi enkelt gjøre videre, og si at en streng 3-kategori er en kategori der morfiene mellom to objekter danner en streng 2-kategori, og ved å følge denne logikken kan vi enkelt lage strenge n-kategorier. 
Et problem med dette er at det er ``mer av det samme''. 
Vi har altså laget kategorier bare med ekstra trinn. 
Det funker godt i visse tilfeller, men ikke i alle vi vil. 
Et eksempel er i homotopiteori. 
Vi kunne ønsket oss å lage en streng 2-kategori med topologiske rom som 0-celler, kontinuerlige funksjoner som 1-celler og homotopier mellom disse som 2-celler, men komposisjon av homotopier er ikke assosiativt, så mengden av kontinuerlige funksjoner mellom to topologiske rom og homotopier mellom de danner ikke en kategori. 
Dermed danner hele strukturen ikke en streng 2-kategori. 
Vi kan løse dette ved å se på homotopiklasser av homotopier istedenfor, noe som vil gjøre komposisjonen assosiativ. 
For å gjøre dette må vi definere homotopier mellom homotopier, så vi kunne kanskje ønske oss at tilsammen så danner alle disse fire typene celler en streng 3-kategori, men vi får samme problem med assosiativiteten i den høyeste dimensjonen. 
Vi må altså fortsette ut i det uendelige for å bli fornøyde. 

En annen løsning er å fjerne ordet ``streng'' for å tillate disse operasjonene som ikke helt er assosiative. 
Dette krever en del arbeid som vi skal se litt på.


\subsection{Definisjoner}

Den første tingen vi kanskje prøver er å gjøre akkuratt det samme vi gjorde for strenge 2-kategorier og se om vi klarer å gjøre definisjonen svakere. 
Dette fører til at vi ikke kan la operasjonene være assosiative, men at vi krever en celle av dimensjon en over for å binde sammen de to ulike måtene å kombinere tre morfier på. 
Denne tankegangen fører til det vi kaller bikategorier. 

\begin{definition}{Bikategori}
    En bikategori består av 
    \begin{itemize}
        \item En samling objekter, ofte kalt 0-celler
        \item For hvert par av 0-celler en kategori $B(x, y)$ hvor objektene kalles 1-celler og morfiene for 2-celler
        \item for hver 0-celle en 1-celle $1_x \in B(x,x)$ kalt identitets 1-cellen ved $x$. 
        \item for hver trippel med 0-celler en funktor $C_{x,y,z}:B(x,y)\times B(y,z)\longrightarrow B(x,z)$ som sender par av 1-celler $(f, g)\mapsto g\circ f$, kalt horisontal komposisjon, og par av 2-celler $(\alpha, \beta ) \mapsto \beta \ast \alpha$ kalt vertikal komposisjon
        \item for hver par av 0-celler har vi isomorfier $l:f\circ 1_x \cong f$ og $r:1_y\circ f\cong f$ kalt venstre og høyre unitor
        \item for hver kvadruppel av 0-celler har vi en funktor $\alpha : B(x, y)\times B(y, z) \times B(z, w) \longrightarrow B(x, w)$ kalt assosiatoren
    \end{itemize}
    slik at følgende diagrammer kommuterer
    
    \begin{center}
        \begin{tikzcd}
            (g\circ 1_y) \circ f \arrow[rd, "r\ast 1_f"'] \arrow[rr, "\alpha"] &          & g\circ (1_y \circ f) \arrow[ld, "1_g\ast l"] \\
                                                                   & g\circ f & 
        \end{tikzcd}
        
        \begin{tikzcd}
                                & ((ih)g)f \arrow[ld, "\alpha"'] \arrow[rr, "\alpha \ast id_f"] &          & (i(hg))f \arrow[rd, "\alpha"] &                                        \\
            (ih)(gf) \arrow[rrd, "\alpha"'] &                                                               &          &                               & i((hg)f) \arrow[lld, "id_i\ast\alpha"] \\
                                &                                                               & i(h(gf)) &                               &                                       
        \end{tikzcd}
    \end{center}
    kalt triangel og pantagon identitenene respektivt. 
\end{definition}

Vi kan prøve å gjøre dette for høyere dimensjoner, og en tilsvarende fullstendig algebraisk definisjon eksisterer også for både trikategorier og tetrakategorier. 
De blir dog mer og mer involverte, kompliserte og trenger større og flere diagrammer. 
Det er en slags konsensus at det ikke er noe vits å skrive ut eksplisitt for $n>4$, da (hvertfall den første) eksplisitte definisjonen på tetrakategorier var \href{https://math.ucr.edu/home/baez/trimble/tetracategories.html}{51 sider lang}. 
Den har blitt  \href{https://arxiv.org/abs/1112.0560}{formalisert og kortet litt ned på} men er fortsatt veldig tungvint å bruke. 
Vi vil jo nå en generell n-kategori, og egentlig $\omega$-kategorier, så denne strategien funker ikke. 
Det vi må gjøre er å introdusere det vi kaller modeller. 

\subsection{Modeller}
Modeller er objekter vi kan konstruere via teori vi allerede kjenner som oppfører seg slik vi ønsker at svake høyere kategorier skal oppføre seg. 
Grunnet at det er mange veier til rom har dette skapt et landskap av ulike definisjoner som kan brukes i ulike tilfeller. 
Krykken med dette er at man ikke vet om objektene er like, eller at de tilsvarer hverandre på en fin måte. 
Mer eksplisitt vet vi ikke om de genererer ekvivalente kategorier. 

Et mulig forsøk på å lage slike modeller kan gjøres ved å bruke spesielle typer mengder, slik som simplisielle mengder eller Opetopiske mengder. 
Man kan også bruke monader eller operader. 
De fleste er veldig tekniske og er langt utenfor hva vi skal gjennom i dag. 
For å nevne noen navn man kan lese senere hvis man er interessert så kan man sjekke ut Segalrom, Segalkategorier, Globulære operader, multisimplisielle mengder eller Trimblekategorier.

En av de mest brukte er såkalte svake komplisielle mengder, en konstruksjon av Ross Street. 
Definisjonen bruker simplisielle mengder med ekstra struktur for å skape en modell for svake høyere kategorier helt opp til uendelig dimensjon. 
Street bruker mye komplisert teori for å skape objekter som definerer svake kategorier av alle dimensjoner. 
Dette krever mye tid, og mer enn jeg kan om teorien for å gjennomføres. 
Men man kan altså gjøre dette opp til uendelig, noe som produserer det vi kaller complisielle $\omega$-kategorier. 

For å gjøre det litt enklere for oss selv ser vi på den mest brukte definisjonen av en $\infty$-kategori, isteden for å bry oss om de veldig generelle svake høyere kategoriene. 
Definisjonen vi skal se på er selvsagt kvasikategorier, og for å forstå definisjonen trenger vi først å forstå simplisielle mengder. 

\section{Simplisielle mengder}

\begin{definition}
    Simplekskategorien $\Delta$ er kategorien bestående av totalt ordnede mengder med svakt ordensbevarende funksjoner som morfier. Objektene skrives ofte som $[n]=\{ 0,1,2,\ldots, n\}$. 
\end{definition}

\begin{definition}
    En simplisiell mengde er en funktor $X:\Delta^{op}\longrightarrow Set$. Vi skriver som regel $X_n$ istedenfor $X([n])$.  
\end{definition}

Kategorien $\Delta$ er generert av to viktige klasser med morfier kalt sideavbildninger og degenerasjonsavbildninger. 
Sideavbildningene defineres ved at det kun er en ordensbevarende injeksjon $[n-1]\longrightarrow [n]$ som ikke treffer tallet $i$. 
Altså har vi for alle $n$ og for enhver $i$ en morfi $\delta_{n,i}:[n-1]\longrightarrow [n]$. 
Bildet av denne morfien i en simplisiell mengde skrives $d_{n,i}:X_{n-1}\longrightarrow X_n$ og kalles den i'te sideavbildningen av $X_n$. 

Tilsvarende er det kun en ordensbevarende surjeksjon $\sigma_{n,i}:[n]\longrightarrow [n-1]$ som tallet $i$ to ganger. 
På lik måte definerer disse degenerasjonsavbildningene $s_{n,i}:X_n\longrightarrow X_{n-1}$ i en simplisiell mengde. 

De kanskje enkleste simplisielle mengdene kalles det standard n-simplekset, $\Delta^n$. 
Dette får vi ved funktoren $\Delta^n=[-, [n]]$. 
Gjennom Yoneda lemmaet får vi at morfier $\Delta^n \longrightarrow \Delta^m$ korresponderer bijektivt til morfier $[n]\longrightarrow [m]$. 

\begin{definition}
    Det simplisielle hornet $\Lambda_k^n$ er unionen av alle de i'te sidene i $\Delta^n$ utenom den k'te. 
    Mer generelt er et horn i X en avbildning $\Lambda_k^n\longrightarrow X$. 
\end{definition}


Vi minner oss selv på litt motivasjon for hva vi prøver å gjøre. 
Vår motivasjon kommer fra eksemplet der ting gikk litt feil tidligere, nemlig topologiske rom. 
Der hadde vi objekter, morfier og så homotopier av morfier. 
Homotopi er en symmetrisk relasjon, og komposisjonen av en homotopi, og så homotopien baklengs er igjen homotopy til identitetshomotopien. 
Så å gå homotopien baklengs fungerer som å ha en invers, hvertfall opp til homotopi. Denne nye høyere homotopien er igjen inverterbar opp til en ny homotopi etc.

Så vi følger dette og ønsker oss en versjon at $\infty$-kategorier der alle k-cellene for $k\geq 2$ skal være inverterbare, eller hvertfall inverterbare opp til en inverterbar $k+1$-morfi. 
Dette er det objektet vi ville frem til gjennom hele foredraget. 

\section{Kvasikategorier}

Endelig har vi kommet frem, og kan nå forstå definisjonen vi startet med, nemlig definisjonen av en kvasikategori. 

\begin{definition}
    En kvasikategori er en simplisiell mengde $X$ slik at alle indre horn har en fylling.
\end{definition}

Vi vet nå hva en simplisiell mengde $X$ er, og hva et horn i $X$ er. 
Et indre horn betyr bare at vi kun ser på $\Lambda_k^n$m for $0<k<n$, alstå ikke ytterkantene. 
At hornet har en fylling betyr at for et horn $\Lambda_k^n\longrightarrow X$ har vi en avbildning $\Delta_k^n\longrightarrow X$ slik at 
\begin{center}
    \begin{tikzcd}
        \Lambda_k^n \arrow[r] \arrow[d, hook] & X \\
        \Delta_k^n \arrow[ru, dotted]         &  
    \end{tikzcd}
\end{center}
kommuterer. 

Så hva betyr dette for oss når vi vil gjøre kategoriteori? 
At vi baserer oss på simplisielle mengder gjør at vi allerede har valgt ut en mengde vi kaller objekter og en mengde morfier mellom de. 
Hornene i den simplisielle mengden vår representerer komponerbare morfier, og fyllingsegenskapen sier at slik komposisjon eksisterer. 
Dette blir veldig eksplisitt når vi lar $n=2$. Da har vi kun ett indre horn i $X$
\begin{center}
    \begin{tikzcd}
        & x_1 \arrow[rd, "f"] &     \\
        x_0 \arrow[ru, "g"] &                & x_2
    \end{tikzcd}
\end{center}
som da representerer to morfier der kodomenet til den ene er domenet til den andre, altså det vi ville kalt komponerbare morfier i en kategori. 
At en fylling eksisterer betyr at det finnes en tredje morfi 
\begin{center}
    \begin{tikzcd}
        & x_1 \arrow[rd, "g"] &     \\
        x_0 \arrow[ru, "f"] \arrow[rr] &                     & x_2
    \end{tikzcd}
\end{center}
som fullfører diagrammet, altså at det eksisterer en morfi vi kan tenke på som komposisjonen av disse to komponerbare morfiene. 
Denne trenger ikke være unik, så det kan eksistere mange ulike måter å komponere morfier, noe vi ønsket fra eksempelet med kategorien av topologiske rom som en høyere kategori tidligere. 

Vi ser at det er smart å ikke ha med de ytre hornene, da foreksempel fyllingen av hornet 
\begin{center}
    \begin{tikzcd}
                     & x_1 \arrow[rd, "g"] &     \\
        x_2 \arrow[rr, "1"'] &                     & x_2
    \end{tikzcd}
\end{center}
impliserer eksistens av inverser, noe vi ikke ønsker å kreve generelt. 

For å hoppe tilbake til den definisjonen av en kategori vi ikke så på i starten, så er denne egenskapen vi krevde at alle indre horn har en unik fylling. 
Altså er en kvasikategori der alle fyllingene er unike en måte å definere en (liten) kategori på! 
Denne delen av teorien er også kul og inkluderer ting som nerver til kategorier og realiseringsfunktorer, men dette spares til en annen anledning.